 \cleardoublepage
\chapter{Conclusion}

\section{Summary}

With the use of SDN and OpenFlow a framework for creating programmable networks already exist. However the integration and use of its full extent within OpenStack is not given today, without the integration of a SDN Controller. The network bandwidth supplied to different servers is on a best-effort basis and topologies that are deployed using the orchestration service Heat can not be placed on specific hosts. With this work the first step for those main objectives are made. With the help of the Open vSwitch configuration tool it is possible to get information about the current state of the switched network and enable QoS using Queues. The required configuration and information can now be accessed through an API. The Elastic Media Manager is able to access this information and make decisions about where the topology has the best-performing network connectivity. The bandwidth rates according to their Service-Level Agreements can be easily updated and extended. The evaluation shows step-by-step the improvements in bandwidth rates with the use of the Connectivity Manager in relation to the requirements.

\section{Problems Encountered}

One big problem during the development process was getting a stable testbed running with the correct configuration. Devstack was a big help in this process, because other than adjusting the configuration file not many manual corrections needed to be made. However some of the features that are in the 'stable' version of OpenStack Juno still need further bug-fixing and can't be used in a production environment. During the tests of the various state-of-the-art solutions a lot of time was spent on testing different versions, which was unsuccessful. Due to this fact, comparisons to other solutions for enabling Quality of Service could not be evaluated. 

\section{Future Work}

For future development the algorithm for selecting the best-performing host could take more dynamic factors into account. One possibility is to check the network-speed of all the host's network interfaces. Furthermore it would be a good approach to implement QoS as a Neutron extension and make use of the already existing Neutron API. This implementation could be based on the already existing blueprint and be shared with the OpenStack community to achieve an integration into the Neutron repository. A further step that could be evaluated for QoS is to add new Flow entries for the rate-limited Queues.