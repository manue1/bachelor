\chapter{Introduction}
\label{chapter_introduction}



\section{Motivation}

The demands on networks have changed dramatically in the past two decades, with an ever-growing number of people and devices relying on interconnected applications and services. The underlying infrastructure has been left mostly unchanged and is approaching its limits. In order to resolve this, Software Defined Networking (SDN) is going to be extending and replacing parts of traditional networking infrastructures. SDN separates the network into control and forwarding planes and therefore allows a more efficient orchestration and automation of network services.

The use of cloud-based services, with not only competitive pricing but also high-availability and fast network access,  is taking over the traditional self-hosted data centers. The ease of administration and the deployment of new Virtual Machines (VMs) on the fly make it possible to create a Topology of Computers with no effort.

Network services have different requirements, depending on the type of data and their importance. The classification of network traffic can be done through Quality of Service (QoS). A new approach has to be made to enable the use of QoS in virtualized cloud infrastructures like OpenStack, to achieve controlled traffic from the deployment of Virtual Machines on.


\section{Network Architecture}
Today's traffic patterns, the rise of cloud computing and "big data" to only name a few examples, are exceeding the capacity of classic network architectures. With scalable computing and storage the common-place tree-structured network infrastructure with Ethernet switches are not efficient and manageable enough. 

The increasing complexity of problems that have to be faced in networks and the need to control network traffic through software, are only a selection of the reasons why the Open Networking Foundation (ONF) developed an approach called Software-Defined Networking (SDN).

SDN is a leading-edge approach where the network control is separated from the forwarding functions. The centralized network intelligence allows programming the network, without a need to access the underlying infrastructure. Therefore a shift of today's networks to more flexibility, programmability and scalability is going to take place.

\section{Objective}

\section{Scope}
In virtualized cloud infrastructure like OpenStack, the placement of Virtual Machines (VMs) on a particular compute node can be decided on by comparing different run-time parameters. The network connectivity between those VMs has to be prioritized and classified into different classes, depending on the service that are running on it.

Currently there are a number of solutions for managing network connectivity between VMs. A comparison and their current limitations follows in the next section. The chosen approach is to extend the existing network control and management services with Quality of Service (QoS) capabilities. In support of the thesis the Connectivity Manager will be implemented and the differences in bandwidth usage will be shown in one use-case.
\section{Overview}
