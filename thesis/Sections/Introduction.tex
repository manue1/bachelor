\chapter{Introduction}
\label{chapter_introduction}



\section{Motivation}

The demands on networks have changed dramatically in the past two decades, with an ever-growing number of people and devices relying on interconnected applications and services. The underlying infrastructure has been left mostly unchanged and is approaching its limits. In order to resolve this, Software Defined Networking (SDN) is going to be extending and replacing parts of traditional networking infrastructures. SDN separates the network into control and forwarding planes and therefore allows a more efficient orchestration and automation of network services.

The use of cloud-based services, with not only competitive pricing but also high-availability and fast network access,  is taking over traditional self-hosted data centers. The ease of administration and deployment of new Virtual Machines (VMs) on the fly make it possible to effortlessly create a topology of servers running different services.

Network services have different requirements, depending on the type of data and their importance. The classification of network traffic can be done through Quality of Service (QoS). A new approach has to be made to enable the use of QoS through an API in virtualized cloud infrastructures like OpenStack, to achieve controlled traffic right from the deployment of Virtual Machines on.


\section{Network Architecture}
Today's traffic patterns, the rise of cloud computing and "big data" to only name a few examples, are exceeding the capacity of classic network architectures. With scalable computing and storage the common-place tree-structured network infrastructure with Ethernet switches are not efficient and manageable enough. 

The increasing complexity of problems that have to be faced in networks and the need to control network traffic through software, are only a selection of the reasons why the Open Networking Foundation (ONF) developed an approach called Software-Defined Networking (SDN).

SDN is a leading-edge approach where the network control is separated from the forwarding functions. The centralized network intelligence allows programming the network, without a need to access the underlying infrastructure. Therefore a shift of today's networks to more flexibility, programmability and scalability is going to take place.

\section{Objective}

The primary objective of this work is the development of a network orchestrator which is able to apply Quality of Service to the network interfaces of Virtual Machines. These Virtual Machines are deployed with OpenStack Nova and connected to an Open vSwitch, which uses OpenFlow. Another task of the Connectivity Manager is to select which OpenStack hypervisor new VMs should be running on, which takes different runtime parameters into account. The CM should be able to be applied in environments with scalable hypervisors and VMs.

\section{Scope}
The scope of this work includes a Connectivity Manager which will have a Connectivity Manager Agent running on the cloud controller within the OpenStack infrastructure, to provide access to the hypervisors of OpenStack Nova. These two components have to be implemented and integrated with the existing Open vSwitches. As a reference for a cloud infrastructure, multimedia communications like the Nubomedia project will be used. The deployment of this cloud is then tested on different performance characteristics like network bandwidth, latency, CPU utilization and memory usage.

In virtualized cloud infrastructure like OpenStack, the placement of Virtual Machines (VMs) on a particular compute node can be decided on by comparing different run-time parameters. The network connectivity between those VMs has to be prioritized and classified into different classes, depending on the service that are running on it.

Currently there are a number of solutions for managing network connectivity between VMs. A comparison and their current limitations follows in the next section. The chosen approach is to extend the existing network control and management services with Quality of Service (QoS) capabilities. In support of the thesis the Connectivity Manager will be implemented and the differences in bandwidth usage will be shown in one use-case.

\section{Overview}

\textbf{Chapter 1} begins with the motivation for this thesis and gives a brief introduction into the objectives and the scope.

\textbf{Chapter 2} gives an overview of traditional network concepts and a introduction to SDN and its components. Furthermore the different services that make up OpenStack will be described.

\textbf{Chapter 3} conceptualizes the state-of-the-art solutions that are currently available and evaluates their implementation and limitations.

\textbf{Chapter 4} contains an analysis of requirements and an architectural overview of the Connectivity Manager. Moreover design aspects are introduced and illustrated according to their requirements.

\textbf{Chapter 5} examines the implementation of the Connectivity Manager and Agent.

\textbf{Chapter 6} evaluates the network performance tests on the basis of a particular use-case.

\textbf{Chapter 7} summarizes the results of this work and gives an overview on possible future work.


