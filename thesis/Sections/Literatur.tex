%F�r das nachfolgende exemplarische Literaturverzeichnis wurde die einfache thebibliography-Umgebung von Latex verwendet. F�r Studien- und Diplomarbeiten mit weniger als 40 Quellen sollte diese auf jeden Fall ausreichen und hat gegen�ber dem komplexen Bibtex-Paket weiterhin den Vorteil flexiblerer Formatierungsm�glichkeiten.
%
%Im Anschluss an das exemplarische Literaturverzeichnis ist ein zweites Verzeichnis beigef�gt, welches weiterf�hrende Quellen zum vorliegenden Latex-Template enth�lt: Download-Links zur Software, freie Online-Latex-Handb�cher usw.



\begin{thebibliography}{Ti}



\bibitem[Abdel-Aziz 71]{abdelaziz71} Y. I. Abdel-Aziz and H. M. Karara, \glqq Direct linear transformation from comparator coordinates into object space coordinates in close-range photogrammetry\grqq, in: Symposium on Close-Range Photogrammetry, issue 11, pp. 1--18, University of Illinois at Urbana-Champaign, 1971.

\bibitem[AutTech 07]{auttech07} Firma Automation Technology GmbH in 22946 Trittau, Produkt�bersicht, Downloads und Datenbl�tter.  URL: \url{http://www.automationtechnology.de}

\bibitem[Dang 06]{dang06} T. Dang, C. Stiller and C. Hoffmann, \glqq Self-calibration for Active Automotive
Stereo Vision\grqq, Proc. of the IEEE Intelligent Vehicles Symposium, pp. 364--369, Japan, Tokyo, 2006.
\end{thebibliography}


