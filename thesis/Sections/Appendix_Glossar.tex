\chapter{Glossar}
\label{anhang_e}



\textcolor{darkred}{Anmerkung: Das vorliegende Glossar wurde ohne die Zuhilfenahme der speziellen Glossarumgebungen von Latex erstellt, um eine etwas freiere Formatierung nutzen zu k�nnen.}
\medskip



\interlinepenalty=10000 % keine Schusterjungen, keine Hurenkinder



\begin{description}

\item[\bf{2,5D-Datensatz}] $\rightarrow$Tiefenbild.

\item[\bf{3D-Modell, 3D-Modellerfassung (optische)}] Der Begriff des 3D-Modells wird in der vorliegenden Arbeit f�r wasserdichte Oberfl�chenmodelle verwendet. Dies dient zur Abgrenzung gegen�ber 3D-Volumenmodellen und $\rightarrow$Tiefenbildern. Der Begriff der Optischen 3D-Modellerfassung umschlie�t hier neben der eigentlichen Sensordatenauswertung auch die $\rightarrow$3D-Registrierung und die Schritte der Nachbearbeitung wie Gl�tten und Ausd�nnen der Daten und Stiching-Operationen.

\item[\bf{3D-Registrierung}] Vgl. Abschnitte 2.4, 4.3 und $\rightarrow$Registrierung.

\end{description}
\interlinepenalty=100



