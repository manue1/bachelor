\chapter{Symbole}\index{Symbole}


\begin{tabular}{lcl}
Symbol 	 & Gr�sse/Einheit 	& Bedeutung \\ \vspace{0.25cm}
$\vec{x_e}$		& - & Einheitsvektor in x-Richtung \\
$\vec{y_e}$		& - & Einheitsvektor in y-Richtung \\
$\vec{z_e}$		& - & Einheitsvektor in z-Richtung \\
 A						& - & Verst�rkung\\
$\alpha_S$ 	  & - &	Seebeck-Koeffizient (Thermospannung)\\
$ \alpha $    & - & R�ckkoppelfaktor\\
$a_n$ 	  		& - & Die Fourierkoeffizienten der $\cos$-Funktion\\
$b_n$ 	  		& - & Die Fourierkoeffizienten der $\sin$-Funktion\\
$\vec{B}$ 	  & - & Magnetische Induktion\\
$c_n$ 	  		& - & Die Fourierkoeffizienten der $e^{j n \omega t} $-Funktion\\
$\vec{D}$ 	  & - & Dielektrische Verschiebung\\
$\vec{E}$ 	  & - & Elektrisches Feld \\
$f(t)$ 	  		& - & Funktion in der Zeitdom�ne\\
$\underline{F}(\omega)$& - &  	  	Fouriertransformierte von $f(t)$ \\
$\vec{H}$ 	  & - &	Magnetische Feldst�rke\\
$I$ 	  			& [A] & Strom\\
$I_0\left(t\right)$ & - & 	  	Besselfunktion \\
$\hat I$ 	  	& - & Amplitude des Stromes\\
$i$ 	  			& - & Stromdichte\\
$J_0\left(t\right)$ &-& 	  	Besselfunktion\\
$j$ 	  			& - & Imagin�re Einheit. Wir verwenden $j$, um eine Verwechslung mit der Stromdichte $i$ zu vermeiden\\
$p$ 	  			& - & Parameter f�r Ortskurven in der komplexen Ebene\\
$Q$ 	  			& - & Ladung \\
$T_p$ 	  		& [s] & Periodendauer\\
$U$ 	  			& [V] & Spannung \\
$\hat U$ 	  	& 		& Amplitude der Spannung\\
$\underline{Y}$ & - &	  	Komplexer Leitwert\\
$\underline{Z}$ &  -  &	  	Komplexe Impedanz\\
$\underline{z}$ &	  - &	Normierte komplexe Impedanz\\
$\delta(t)$ 	  & - & 	Die Diracsche Deltafunktion\\
$\varepsilon _0$ & - &	$8,85\times 10^{-12}{\frac{(As)^2}{Nm^2}}$ 	Dielektrizit�tszahl des Vakuums\\
$\varepsilon_r$  & - & 	  	Dielektrizit�tszahl eines Materials\\
$\rho $ 	  	    & - & Ladungsdichte\\
$\varphi $ 	  	  &[rad], [�] & Phase \\
$\mu_0$ 				& -& $4\pi\times 10^{-7} {\frac{Vs}{Am}}$ 	Induktionskonstante\\
$\mu _r$ 	  	& - & Relative Permeabilit�t\\
$\omega$ 	  	& - & Kreisfrequenz\\
$\omega_0$ 	  	& -& Kreisfrequenz der freien, unged�mpften	Schwingung eines Oszillators\\
$\Omega$ 	  	& -& Normierte Kreisfrequenz\\

\end{tabular}







