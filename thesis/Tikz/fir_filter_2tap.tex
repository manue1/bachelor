% Library for block diagrams and signal flow graphs
% Author: Matthias Hotz

\documentclass{article}

\usepackage{tikz}
\usetikzlibrary{dsp,chains}

\DeclareMathAlphabet{\mathpzc}{OT1}{pzc}{m}{it}
\newcommand{\z}{\mathpzc{z}}

\begin{document}

\centering

\begin{tikzpicture}

	% Place nodes using a matrix
	\matrix[row sep=2.5mm, column sep=5mm]
	{
		%--------------------------------------------------------------------
		\node[dspnodeopen,dsp/label=left]  (m00) {$x[n]$};    &
		\node[coordinate]                  (m01) {};          &
		\node[dspnodefull]                 (m02) {};          &
		\node[dspsquare]                   (m03) {$\z^{-1}$}; &
		\node[dspnodefull]                 (m04) {};          &
		\node[dspsquare]                   (m05) {$\z^{-1}$}; &
		\node[coordinate]                  (m06) {};          &
		\node[coordinate]                  (m0X) {};          \\
		%--------------------------------------------------------------------
		\node[coordinate]                  (m10) {};          &
		\node[coordinate]                  (m11) {};          &
		\node[dspmixer, dsp/label=right]   (m12) {$b_0$};    &
		\node[coordinate]                  (m13) {};          &
		\node[dspmixer, dsp/label=right]   (m14) {$b_1$};    &
		\node[coordinate]                  (m15) {};          &
		\node[dspmixer, dsp/label=right]   (m16) {$b_2$};    &
		\node[coordinate]                  (m17) {};          &
		\node[coordinate]                  (m1X) {};          \\
		%--------------------------------------------------------------------
		\\
		%--------------------------------------------------------------------
		\node[coordinate]                  (m20) {};          &
		\node[coordinate]                  (m21) {};          &
		\node[coordinate]                  (m22) {};          &
		\node[coordinate]                  (m23) {};          &
		\node[dspadder]                    (m24) {};          &
		\node[coordinate]                  (m25) {};          &
		\node[dspadder]                    (m26) {};          &
		\node[coordinate]                  (m27) {};          &
		\node[dspnodeopen,dsp/label=right] (m2X) {$y[n]$};    \\
		%--------------------------------------------------------------------
	};

	% Draw connections
	
	\begin{scope}[start chain]
		\chainin (m00);
		\chainin (m02) [join=by dspflow];
		\chainin (m12) [join=by dspconn];
		\chainin (m22) [join=by dspline];
	\end{scope}

	\foreach \i [evaluate = \i as \j using int(\i+1),
	             evaluate = \i as \k using int(\i+2),] in {2,4}
	{
		\begin{scope}[start chain]
			\chainin (m0\i);
			\chainin (m0\j) [join=by dspconn];
			\chainin (m0\k) [join=by dspline];
			\chainin (m1\k) [join=by dspconn];
			\chainin (m2\k) [join=by dspconn];
		\end{scope}
		\draw[dspconn] (m2\i) -- (m2\k);
	}

	\draw[dspflow] (m26) -- (m2X);

\end{tikzpicture}

\vspace{2em}


\end{document}
